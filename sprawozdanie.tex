\documentclass[11pt,wide]{mwart}

\usepackage[OT4,plmath]{polski}
\usepackage{amsmath,amssymb,amsfonts,amsthm,mathtools}
\usepackage{color}
\usepackage{fontspec}
\usepackage{listings,times}

\usepackage{bbm}
\usepackage[colorlinks=true, urlcolor=blue]{hyperref}
\usepackage{url}

\usepackage{lmodern}\normalfont

\usepackage{multicol}

\newcommand{\w}[1]{
    \indent\textbf{#1}
 }

\newcommand{\Ta}[1]{
  \begin{tabular}{| p{0.9\textwidth} |}
    \hline
    #1
    \hline
  \end{tabular}
 }

\title{Needfinding\\{\large KCK, zadanie 2.}}
\author{Łukasz Czapliński}
\date{\today}

\begin{document}

\thispagestyle{empty}
\maketitle
\pagebreak

\section{Analiza problemu}
Czynnością, której obserwacji się podjąłem, jest proces planowania i wykonywania codziennych zadań. Motywacją do tego były wielokrotnie widziane w internecie tematy poświęcone motywacji. Wydaje się, że wielu osobom brakuje impulsu do podjęcia działań. Chcieliby zacząć ćwiczyć, nauczyć się gry na instrumencie czy danego języka lub po prostu zacząć wynosić śmieci zanim kluczowym do tego zajęcia narzędziem stanie się siekiera. To badanie ma na celu sprawdzenie, czy faktycznie tak jest.
\section{Obserwacje i wywiady}
\subsection{Student prawa}
\subsubsection{Ankieta}
\Ta{Czy istnieją rzeczy, które chciałeś (chciałaś) zacząć robić, ale z różnych przyczyn nie zrobiłeś (zrobiłaś)?\\
\indent Tak.\\
Jak często powodem były zajęcia pokroju siedzenia na facebooku, grania?\\
\indent Komputer i rozleniwienie bardzo częstym powodem.\\
Czy próbowałeś (próbowałaś) się motywować w jakiś sposób? Np granie dopiero po zrobieniu odpowiednich rzeczy?\\
\indent Próbowałem, ale nie dawałem rady. Słaba dyscyplina.\\
Jak myślisz, czy usprawnienie sposobu motywowania mogłoby ci pomóc?\\
\indent Może, nie wiem o jakie usprawnienie chodzi.\\
Czy korzystasz z przypomnień na komórce lub komputerze?\\
\indent Korzystam, normalnie działają – przypominają.\\}
\subsubsection{Wywiad}
\begin{multicols}{2}
\noindent \w{Q:}Uzywasz przypomnien do wiekszych rzeczy, ktore na pewno musisz zrobic, czy tez do bardziej codziennych czynnosci?\\
np sprzatanie\\
\w{A:}Do większych rzeczy. Lekarze, spotkania.\\
Albo!\\
Że MAM O CZYMŚ POMYŚLEĆ.\\
Przypominam sobie o temacie, który muszę sam ze sobą podjąć wtedy i wtedy.\\
\w{Q:}a co z codziennymi czynnosciami?\\
wynies smieci/wyjdz z psem ktorego nie masz/ pozmywaj\\
\w{A:}miałbym to zapisywać na komórce?\\
Lul.\\
Ale zapisuję sobie czasem plan dnia.\\
Z którego nic nie wychodzi.\\
\w{Q:}czemu nic z niego nie wychodzi?\\
\w{A:}Tzn część wychodzi, ale dyscyplina lub okoliczności zazwyczaj nie pozwalają mi się dostosować do niego w pełni.\\
Wychodzi może z 50\%.\\
\w{Q:}hm\\
jesli moglbys latwo modyfikowac plan\\
przesunac sobie jakas czynnosc na za jakis czas\\
to ulatwiloby ci to wypelnianie tego planu?\\
\w{A:}no i zazwyczaj to robię.\\
Odjutryzm.\\
Przed spaniem - planik co do godziny i minuty, ulala, a w czasie dnia korekty.\\
\w{Q:}a jesli dodatkowo dostawalbys jakas motywacje zeby to zrobic teraz?\\
np motywujacy filmik czy cos\\
i kare/nagrode za wykonanie\\
to faktycznie uzywalbys tego czy raczej oszukiwal?\\
\w{A:}Nie wiem.\\
Chyba bym oszukiwał.\\
Niby jaka by to mogła być kara.\\
\w{Q:}moglbys sam ustalac przy zapisywaniu zadania\\
\w{A:}Prawdziwą karą są wydarzenia już zaraz, które zmuszają mnie do roboty. Ale jeśli chodzi o plan "zacząć się uczyć dalej Niemca" to nie mam żądnego sprawdzianu, który byłby karą w postaci pały.\\
\w{Q:}moglbys sobie sam ustawic np, ze komputer przestanie ci na dzien uruchamiac twoja ulubiona gre komputerowa ;d\\
ew narysuje ci karnego kutasa na pulpicie\\
\w{A:}No dobra, ale skąd komputer będzie wiedział, że nie spełniłem zadania.\\
\w{Q:}no i wlasnie ;d\\
pytanie czy bys kilkal "tak, zrobilem" mimo niezrobienia\\
\w{A:}Hah!\\
Nie wiem.\\
Nie klikałbym.\\
Jakbym miał takie coś to bym nie klikał.\\
\end{multicols}
\subsubsection{Obserwacje}
Obserwowany na początku dnia. Był umówiony na spotkanie, nastawił budzik. Wyłączył go przez sen. Dopiero telefon od znajomych przypomniał mu, że powinien wstać. Nie pamiętał jak dojechać na spotkanie - musiał znowu dzwonić na spotkanie. Nie spakował się wcześniej - musiał pakować się przed wyjściem. Nie zdążył się dobrze przygotować: część rzeczy (naładowanie baterii) zajmowała więcej czasu niż miał dostępne.
\subsection{Student informatyki}
\subsubsection{Ankieta}
\Ta{
\noindent Czy istnieją rzeczy, które chciałeś (chciałaś) zacząć robić, ale z różnych przyczyn nie zrobiłeś (zrobiłaś)?\\
\indent Chciałem zacząć przygotowywać się do zajęć na uczelnię, ale z powodu ogromnej niechęci do robienia czegokolwiek konstruktywnego grałem/oglądałem filmy etc.\\
Czy próbowałeś (próbowałaś) się motywować w jakiś sposób? Np granie dopiero po zrobieniu odpowiednich rzeczy?\\
\indent tak, ale dopiero w momencie zbliżającego się terminu, na który miałem coś przygotować. Czasami jednak nie starczało czasu, a czasami nadal nie byłem zmotywowany\\
Jak myślisz, czy usprawnienie sposobu motywowania mogłoby ci pomóc?\\
\indent Usprawnienie sposobu motywowania - nie. Inny sposób motywowania - może.\\
Czy korzystasz z przypomnień na komórce lub komputerze?\\
\indent Nie, nie korzystam z żadnego sposobu przypominania gdyż wszystko pamiętam. Po prostu mi się nie chce.\\}
\subsubsection{Wywiad}
Ankietowany grał w LoLa, brak komunikatywności mimo wcześniejszego umówienia się.
\subsubsection{Obserwacje}
Obserwowany długoterminowo - współlokator. Większość czasu spędza na graniu na komputerze lub oglądaniu filmów. Nie chodzi na większość zajęć - nie chce mu się. Gdyby nie konieczność załatwiania potrzeb nie wychodziłby z pokoju. Ekstremalne problemy z motywowaniem się do czegokolwiek. Jeśli czegoś od niego trzeba, wystarczy odłączyć internet - aby go odzyskać jest w stanie zrobić wszystko. Możliwość motywowania przez program kontrolujący dostęp do gry - musi jednak on być na tyle inteligentny, by nie przerwał mu rozgrywki (wówczas możliwe ofiary śmiertelne).
\subsection{Studentka informatyki}
\subsubsection{Ankieta}
\Ta{
\noindent Czy istnieją rzeczy, które chciałeś (chciałaś) zacząć robić, ale z różnych przyczyn nie zrobiłeś (zrobiłaś)?\\
\indent Oczywiście. Niemal codziennie są takie rzeczy. \\
Jak często powodem były zajęcia pokroju siedzenia na facebooku, grania?\\
\indent Jak mam większe plany to facebook raczej mnie nie zatrzymuje. Jeśli już, to najczęściej chodzi o jakieś obowiązki, których zwyczajnie nie chcę robić.\\
Czy próbowałeś (próbowałaś) się motywować w jakiś sposób? Np granie dopiero po zrobieniu odpowiednich rzeczy?\\
\indent Myślałam kiedyś nad założeniem sobie blokady: 30 minut na facebooku dziennie. Ale tego nie zrobiłam, bo okazało sie, ze to nie to mi przeszkadza, ale moja własna niechęć. \\
Czy to działało?\\
\indent Nie działało. Brałam się za inną głupotę. Zadziałałoby gdyby mnie zamknąć w domu tylko z mopem i niczym więcej.\\
Jak myślisz, czy usprawnienie sposobu motywowania mogłoby ci pomóc?\\
\indent No jak mi ktoś da nagrodę za sprzątanie, to pewnie by mi pomogło.\\
Czy korzystasz z przypomnień na komórce lub komputerze?\\
\indent Nie za często. Raczej zapisuję sobie na kartce. Lubię tę fizyczną mozliwość skreślnia czegoś jak to zrobię.\\}
\subsubsection{Wywiad}
\begin{multicols}{2}
\noindent \w{Q:}czy jakby nagroda nie pochodzila od kogos, ale np jakas aplikacja cie karala/nagradzala
to uzywalabys jej?\\
\w{A:}nie\\
\w{Q:}czemu?\\
\w{A:}automat nie może mieć nade mną władzy\\
\w{Q:}np ustawiasz sobie "1800 wynies smieci. nagroda smieszne obrazki, kara; 50gr"\\
i jakas appka o 1800\\
ci wywala jakis motywujacy filmik\\
i zadanie: wynies smieci\\
to faktycznie wynioslabys smieci\\
i kliknela "zrobione"\\
(wygladaloby to jak skreslanie czegos z listy)\\
czy olala i kliknela "zrobione"\\
czy tez olala i kliknela "niezrobione"?\\
\w{A:}a byłaby opcja drzemka?\\
\w{A:}bo jakbym siedziałą na kiblu np. to bym rozwaliła telefon\\
\w{A:}nie, to by nie działało na mnie\\
nie zawsze mam ochotę na śmieszne obrazki\\
a jakby mi jakiś budzik o 18 powiedział, zę mam wyniesć śmieci, to bym puknęła się w czoło i stwierdziła, że wiadro wcale nie jest pełne\\
\w{Q:}to raczej ty bys ustawiala budzik na 1800 po zauwazeniu rano przy sniadaniu ze jest pelne\\
i by ci przypomnial po powrocie zajec ze przydaloby sie to zrobic\\
\w{A:}to bym wynisła od razu\\
\w{Q:}a nie walnac na lozko\\
zawsze masz czas na to?\\
\w{A:}po co mają śmierdzieć?\\
no kosz mam po drodze\\
\w{Q:}no to zalozmy mniej natychmiastowe czynnosci\\
np zadanie z kck\\
czy cokolwiek innego, co nie zawsze mozesz zrobic od razu\\
\w{A:}nie... u mnie to nie działa\\
ja tak nie umiem\\
ja wszystko robię ad hoc\\
\w{Q:}i nie myslisz nigdy np "kurcze, chcialabym zaczac sie uczyc xxx?"\\
\w{A:}noo eee tak\\
ale wtedy siadam i szukam kursu \\
książki\\
dokonuję odpowiednich przygotowań i leci \\
\w{Q:}zawsze?\\
nie zapominasz o niczym?\\
\w{A:}no.. nie\\
pamiętam kiedy ma urodziny większość z moich znajomych\\
jeżlei zaczynam zapominać to idę spać \\
\end{multicols}
\subsubsection{Obserwacje}
Większość czasu, przez który rzekomo się uczy ogląda strony internetowe luźno powiązane z tematem. Zadania wykonuje jeśli najdzie ją na to ochota.
\section{Spostrzeżenia}
\begin{itemize}
  \item Większość ludzi ma problemy z planowaniem: na mniejszą lub większą skalę.
  \item Na mniejszą: harmonogram dnia.
  \item Na większą: chcą coś zacząć robić regularnie, brakuje im motywacji.
  \item Standardowe rozwiązania są na zbyt małą skalę, przez to mało elastyczne (budzik) lub na zbyt dużą i przez to mało motywujące (notatki na przyszłość).
  \item Ludzie lubią być nagradzani jak coś zrobią. Lubią też, jak ta nagroda wydaje się spontaniczna.
  \item Często też jedyną motywacją jest postawienie ich pod ścianą: mogą zrobić tylko to, co trzeba. Nawet siedzenie i nie robienie niczego wydaje się wtedy interesujące.
\end{itemize}
\section{Inspiracje}
\begin{itemize}
  \item Powiązanie kar i nagród z czynnościami: użytkownik miałby kontrolę (w mniejszym lub większym stopniu) na konsekwencje wykonania zadania.
  \item System znany z gier komputerowych: można wprowadzić porównywanie się ze znajomymi.
  \item Weryfikacja zadań mogłaby się odbywać przez serwis społecznościowy: inni potwierdzaliby wykonanie zadania
  \item Weryfikacją codziennych zadań mógłaby zająć się aplikacja mobilna powiązana z systemem NFC: np naklejka na koszu na śmieci pozwalałaby na weryfkację, że śmieci zostały wyniesione.
  \item Plan zajęć powinien jedynie sugerować czasy, nie sztywno je określać (jak budzik). System drag`n`drop mógłby pozwolić na łatwe modyfikowanie kolejności i czasu zadań.
\end{itemize}
\end{document}
